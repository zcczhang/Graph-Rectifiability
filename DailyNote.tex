\documentclass[12pt, a4paper]{article}


\usepackage{hyperref}
\hypersetup{
    colorlinks,
    citecolor=blue,
    filecolor=blue,
    linkcolor=blue,
    urlcolor=blue
}

\usepackage{amsmath}
\DeclareMathOperator*\lowlim{\underline{lim}}
\DeclareMathOperator*\uplim{\overline{lim}}
% \usepackage{amsthm}
% \theoremstyle{definition}
\newtheorem{definition}{Definition}[subsection]
\newtheorem{theorem}{Theorem}[subsection]
\newtheorem{corollary}{Corollary}[subsection]
\newtheorem{exercise}{Exercise}
\newtheorem{property}{Property}[subsection]
\newtheorem{proposition}{Proposition}[subsection]
\usepackage{graphicx}
\usepackage{amsfonts}
\usepackage[letterpaper,margin=1in]{geometry}
\usepackage{float}
\usepackage{enumerate} 


\newtheorem{innercustomgeneric}{\customgenericname}
\providecommand{\customgenericname}{}
\newcommand{\newcustomtheorem}[2]{%
  \newenvironment{#1}[1]
  {%
   \renewcommand\customgenericname{#2}%
   \renewcommand\theinnercustomgeneric{##1}%
   \innercustomgeneric
  }
  {\endinnercustomgeneric}
}

\newcustomtheorem{customexercise}{Exercise}
\newcustomtheorem{customsol}{Solution}


\title{\textbf{Graph Rectifiability Summer Research} \\ [0.5cm] \sffamily{Daily Report\\(in reverse chronological order)}}

\author{\textbf{\textit{Charles Zhang}}}

\date{Summer 2021}

\begin{document}

\maketitle

{
    \hypersetup{linkcolor=black}
    \tableofcontents
}

\newpage
\section{May 4}
\subsection{Exercises and Solutions}

\begin{customexercise}{2.1}
    Verify directly from the definitions that Equivalent definitions \ref{bcd-def}(ii) and (iv) give the same values for box dimension.
\end{customexercise}

\begin{customsol}{2.1}
    Let $F$ be a subset of $\mathbb{R}^{n}$, let $N_{\delta}(F)$ denote the smallest number of closed balls of radius $\delta$ that cover $F$ and let $N_{\delta}^{\prime}(F)$ denote the number of $\delta$-mesh cubes that intersect $F$. 
    Consider that the closed ball of radius $\delta\sqrt{n}$ will definitely contains the $\delta$-mesh cube when centers are the same. On the other hand, any closed ball of radius $\delta$ is intersect(and contained if center of the ball is the center of all cubes) at most $4^n$ $\delta$-mesh cubes. Thus, 
    $$
    N_{\delta \sqrt{n}}(F) \leq N_{\delta}^{\prime}(F) \leq 4^{n} N_{\delta}(F)
    $$
    Combining this inequality and dividing by $-\log \delta$,
$$
\frac{\log N_{\delta \sqrt{n}}(F)}{-\log (\delta \sqrt{n})+\log \sqrt{n}} \leq \frac{\log N_{\delta}^{\prime}(F)}{-\log \delta} \leq \frac{\log 4^{n}+\log N_{\delta}(F)}{-\log \delta}
$$
so taking lower limits as $\delta \rightarrow 0$ where $\delta\sqrt{n}\rightarrow 0 $ as well,
$$
\lowlim _{\delta \rightarrow 0} \frac{\log N_{\delta}(F)}{-\log \delta} \leq \lowlim _{\delta \rightarrow 0} \frac{\log N_{\delta}^{\prime}(F)}{-\log \delta} \leq \lowlim _{\delta \rightarrow 0} \frac{\log N_{\delta}(F)}{-\log \delta},
$$
with the other terms disappearing in the limit. Thus, the definition of lower box dimension is the same working with either $N_{\delta}(F)$ or $N_{\delta}^{\prime}(F)$. Taking upper limits, we get a similar conclusion for upper box dimension.
\end{customsol}


\begin{customexercise}{2.2}
    Generalise Proposition \ref{prop2.5} by showing that if $f: F \rightarrow \mathbb{R}^{n}$ satisfies the Hölder condition $|f(x)-f(y)| \leq c|x-y|^{\alpha}$ where $c>0$ and $0<\alpha \leq 1$, then $\underline{\operatorname{dim}_{B}} f(F) \leq(1 / \alpha) \underline{\operatorname{dim}}_{\mathrm{B}} F$ and $\overline{\operatorname{dim}}_{\mathrm{B}} f(F) \leq(1 / \alpha) \overline{\operatorname{dim}}_{\mathrm{B}} F$.
\end{customexercise}

\begin{customsol}{2.2}
    Note that if $\{U_i\}$ is a $\delta$-cover of $F$, then so $\{U_i\cap F\}$. Then, as
    $$
    |f(U_i\cap F)|\leq c|U_i\cap F|^\alpha\leq c|U_i|^\alpha\leq c\delta^\alpha
    $$
    (this can be understood as taking $x, y\in U_i\cap F$ or $x, y\in U_i$ where $|x-y|<\delta$ by construction)
    s.t. $\{f(U_i\cap F)\}$ is a $c\delta^\alpha$ cover of $f(F)$, and hence $N_{c\delta^\alpha}(f(F)) \leq N_\delta(F), \forall \delta>0$(consider $f$ is injective, or, we may have a better cover when considered overall for the image). Thus,
    $$
    \begin{aligned} \underline{\operatorname{dim}}_{B} f(F) &=\varliminf_{c \delta^{\alpha} \rightarrow 0} \frac{\log N_{c \delta^{\alpha}}(f(F))}{-\log c \delta^{\alpha}} \\
        &\leq \varliminf_{\delta \rightarrow 0} \frac{\log N_{\delta}(F)}{-\alpha \log \delta-\log c} \\ &=\frac{1}{\alpha} \varliminf_{\delta \rightarrow 0} \frac{\log N_{\delta}(F)}{-\log \delta}\\
        &=\frac{1}{\alpha} \underline{\operatorname{dim}}_{B} F \end{aligned}
    $$
    and same process can be applied for upper box-counting dimension. 
\end{customsol}


\begin{customexercise}{2.4}\label{calculation}
    Verify that the Cantor dust depicted in Figure \ref{fig:cantordust}, has box dimension 1 (take $E_{0}$ to have side length 1).
    \begin{figure}[H]
        \centering
        \includegraphics[width=.66\textwidth]{images/cantordust.png}
        \caption{ Construction of a ‘Cantor dust’ }
        \label{fig:cantordust}
    \end{figure}
\end{customexercise}

\begin{customsol}{2.4}
    $k^{\text{th}}$ stage of the construction consists of $4^k$ squares of side length $4^{-k}$. Thus, if $4^{-k} < \delta \leq 4^{-k+1}$, the $4^k$ squares of $E_k$ give a $\delta$ cover of $F$, so $N_\delta(F)\leq 4^k$. Then:
    $$
    \overline{\operatorname{dim}}_{\mathrm{B}} F = \uplim_{\delta\rightarrow 0} \frac{\log N_\delta(F)}{-\log \delta}\leq \uplim_{k\rightarrow\infty} \frac{\log 4^k}{-\log 4^{-k+1}} = 1
    $$
    On the other hand, for $4^{-k-1}\leq \delta < 4^{-k}$, the cube can intersect at most two of the squares of $E_k$. There are $4^k$ squares in $E_k$, all containing points of $F$, so at least $4^k/2$ squares of side $\delta$ are required to cover $F$. Then, $N_\delta (F) \geq 4^{k}/2$, so:
    $$
    \underline{\operatorname{dim}_{\mathrm{B}}} F = \lowlim_{\delta\rightarrow 0}\frac{\log N_\delta(F)}{-\log \delta} \geq \lowlim_{k\rightarrow\infty} \frac{\log 4^k/2}{-\log 4^{-k-1}} = 1
    $$

    Therefore, the box-counting dimension of Cantor dust is 1. 
\end{customsol}



\begin{customexercise}{2.5}
    Use Equivalent definition \ref{bcd-def}(i) to check that the upper box dimension of the von Koch curve(shown in Figure \ref{fig:kochcurve}) is at most $\log 4 / \log 3$ and \ref{bcd-def}(v) to check that the lower box dimension is at least this value.
    \begin{figure}[t]
        \centering
        \includegraphics[width=.4\textwidth]{images/Kochcurve.png}
        \caption{(a) Construction of the von Koch curve $F$. At each stage, the middle third of each interval is replaced by the other two sides of an equilateral triangle. (b) Three von Koch curves fitted together to form a snowflake curve.}
        \label{fig:kochcurve}
    \end{figure}
\end{customexercise}

\begin{customsol}{2.5}
    Let $\delta_k = 3^{-k}$, for $E_k$, there are $4^k$ line segments so taking any plane set with diameter at most $\delta_k$ centered at the midpoint of each line segment can cover all points in $F$ so $N_{\delta_k}(F)\leq 4^k$. Then,
    $$
    \overline{\operatorname{dim}}_\mathrm{B}F = \uplim_{k\rightarrow\infty} \frac{\log N_{\delta_k(F)}}{-\log \delta_k}\leq \uplim_{k\rightarrow\infty}\frac{\log 4^k}{\log 3^k} = \frac{\log 4}{\log 3}
    $$
    On the other hand, there are $4^k+1$ vertices of $E_k$ and if we take each vertices as centers of balls with radius $\delta_k = 3^{-k}/2$(it is also sufficient to take $\delta_k = 3^{-k-1}$), there would be at least $4^k+1$ disjoint balls of radius $\delta_k$ with centers in $F$. Then,
    $$
    \begin{aligned}
        \underline{\operatorname{dim}}_{B} F &=\lowlim_{k \rightarrow \infty} \frac{\log N_{\delta_{k}}(F)}{-\log \delta_{k}} \\
        & \geq \lowlim_{k \rightarrow \infty} \frac{\log \left(4^{k}+1\right)}{\log 3^{k+1}} \\
        & \geq \lowlim_{k \rightarrow \infty} \frac{\log \left(4^{k}\right)}{\log 3^{k+1}} \\
        & = \lowlim_{k \rightarrow \infty} \frac{k \log 4}{(k+1) \log 3}\\
        & =\frac{\log 4}{\log 3}
        \end{aligned}
    $$
\end{customsol}




\newpage
\section{May 2-3}
\subsection{Box-Counting Dimensions}

\begin{definition}[$\delta$-mesh]
    The family of cubes of the form
$$
\left[m_{1} \delta,\left(m_{1}+1\right) \delta\right] \times \cdots \times\left[m_{n} \delta,\left(m_{n}+1\right) \delta\right],
$$
where $m_{1}, \ldots, m_{n}$ are integers, is called the $\delta$-mesh or $\delta$-grid of $\mathbb{R}^n$.
\end{definition}

\begin{definition}[Diameter]
    any subset of $n-$dimensional Euclidean space, $\mathbb{R}^n$, the \textbf{diameter} of $U$ is defined as $|U|=\sup \{|x-y|:x, y\in U\}$
\end{definition}

\begin{definition}[Box-Counting Dimension]\label{bcd-def}
    The lower and upper box-counting dimensions of a subset $F$ of $\mathbb{R}^{n}$ are given by
    $$\underline{\operatorname{dim}_{\mathrm{B}}} F=\varliminf_{\delta \rightarrow 0} \frac{\log N_{\delta}(F)}{-\log \delta}$$
    $$\overline{\operatorname{dim}}_{\mathrm{B}} F=\varlimsup_{\delta \rightarrow 0} \frac{\log N_{\delta}(F)}{-\log \delta}$$
    
and the box-counting dimension of $F$ by
$$
\operatorname{dim}_{\mathrm{B}} F=\lim _{\delta \rightarrow 0} \frac{\log N_{\delta}(F)}{-\log \delta}
$$
(if this limit exists), where $N_{\delta}(F)$ is any of the following:
\begin{enumerate}[i]
    \item the smallest number of sets of diameter at most $\delta$ that cover $F$
    \item  the smallest number of closed balls of radius $\delta$ that cover $F$
    \item  the smallest number of cubes of side $\delta$ that cover $F$;
    \item  the number of $\delta$-mesh cubes that intersect $F$;
    \item  the largest number of disjoint balls of radius $\delta$ with centres in $F$  
\end{enumerate}  
Figure \ref{fig:bcd-def-vis} illustrates Five ways of finding the box dimension of $F$.

\end{definition}


\begin{figure}[t]
    \centering
    \includegraphics[width=.66\textwidth]{images/bcd-def-vis.png}
    \caption{Five ways of finding the box dimension of $F$}
    \label{fig:bcd-def-vis}
\end{figure}


\textbf{Motivation:}
If $N_\delta(F)$ obeys, at least approximately, a power law
$$N_{\delta}(F) \simeq c \delta^{-s}$$
for positive constant $c$ and $s$, we say that $F$ has box dimension $s$.
\\
Algorithms to solve for $s$:
$$
\log N_{\delta}(F) \simeq \log c-s \log \delta
$$
so
$$
s \simeq \frac{\log N_{\delta}(F)}{-\log \delta}+\frac{\log c}{\log \delta}
$$
and we might hope to obtain $s$ as
$$
s=\lim _{\delta \rightarrow 0} \frac{\log N_{\delta}(F)}{-\log \delta},
$$
with the second term disapperaing in the limit. And this implies that we assume that $\delta$ is sufficiently small. 


Roughly speaking, equation for dimension says that $N_{\delta}(F) \simeq c \delta^{-s}$ for small $\delta$, where $s=\operatorname{dim}_{\mathrm{B}} F$, or, more precisely, that
$$
N_{\delta}(F) \delta^{s} \rightarrow \infty \quad \text { if } s<\operatorname{dim}_{\mathrm{B}} F
$$
and
$$
N_{\delta}(F) \delta^{s} \rightarrow 0 \quad \text { if } s>\operatorname{dim}_{\mathrm{B}} F .
$$

\textbf{Proof of Equivalence of Definition \ref{bcd-def}}:\\
e.g.: (i)$\Leftrightarrow$(iv): Let $N_\delta(F)$ for the smallest number of sets of diameter $\delta$ that can cover $F$ whereas $N_\delta^\prime(F)$ be the number of $\delta$-mesh cubes that intersect $F$. 
Note that these cubes provide a collection of $N_\delta^\prime(F)$ sets of diameter $\delta\sqrt{n}$(diagonal for "diameter at most $\delta$") that cover $F$, $$N_{\delta\sqrt{n}}(F)\leq N_\delta^\prime(F)$$
On the other hand, set of diameter at most $\delta$ is(or must be) contained in $3^n$ mesh cubes of side $\delta$(e.g. choosing a cube containing some points of the set with its all neighbouring cubes, like the middle one and its 8 neighnors when $n=2$). Then we have:
$$ N_\delta^\prime(F) \leq 3^n N_\delta(F)$$
Combining these inequalities and dividing by $-\log \delta$,
$$
\frac{\log N_{\delta \sqrt{n}}(F)}{-\log (\delta \sqrt{n})+\log \sqrt{n}} \leq \frac{\log N_{\delta}^{\prime}(F)}{-\log \delta} \leq \frac{\log 3^{n}+\log N_{\delta}(F)}{-\log \delta}
$$
so taking lower limits as $\delta \rightarrow 0$,
$$\varliminf_{\delta \rightarrow 0} \frac{\log N_{\delta}(F)}{-\log \delta} \leq \varliminf_{\delta \rightarrow 0} \frac{\log N_{\delta}^{\prime}(F)}{-\log \delta} \leq \varliminf_{\delta \rightarrow 0} \frac{\log N_{\delta}(F)}{-\log \delta}$$
with the other terms diappearing in the limit. Thus, lower limit works as sama as for both $N_\delta(F)$, and similar for upper box dimension. 

\textbf{\textit{\\Calculations}}: See Exercise \ref{calculation}

\textbf{\\Note: }More generally a set $F$ made up of $m$ similar disjoint copies of itself at scale $r$ has $$\operatorname{dim}_{\mathrm{B}} F=\log m /-\log r$$

Consider the neighborhood: $F_{\delta}=\left\{x \in \mathbb{R}^{n}:|x-y| \leq \delta \quad\right.$ for some $\left.y \in F\right\}$. The rate at which the $n$-dimensional volume, that is, $n$-dimensional Lebesgue measure $\mathcal{L}^n$, of $F_\delta$ shrinks as $\delta\rightarrow 0$

This idea extends to fractional dimensions. If $F$ is a subset of $\mathbb{R}^{\prime}$ and $\displaystyle\lim _{\delta \rightarrow 0}\left(\mathcal{L}^{n}\left(F_{\delta}\right) / \delta^{n-s}\right)=c$ for some $s>0$ and $0<c<\infty$, it makes sense to
regard $F$ as $s$ -dimensional, and it turns out that $s$ is just the boxcounting dimension. The number $c$ is called the \textit{$s$-dimensional Minkowski content} of $F$ --- a quantity that is useful in some concepts but has the disadvantages that it does not exist for many standard fractals and that it is not necessarily additive on disjoint subsets, that is, is not a measure. Even if this limit does not exist, we can take lower and upper limits, and these are related to the box dimensions.


\begin{proposition}
    If $F$ is a subset of $\mathbb{R}^{n}$, then
$$
\underline{\operatorname{dim}}_{\mathrm{B}} F=n-\varlimsup_{\delta \rightarrow 0} \frac{\log \mathcal{L}^{n}\left(F_{\delta}\right)}{\log \delta}
$$
$$
\overline{\operatorname{dim}}_{\mathrm{B}} F=n-\lowlim_{\delta \rightarrow 0} \frac{\log \mathcal{L}^{n}\left(F_{\delta}\right)}{\log \delta}
$$
where $F_{\delta}$ is the $\delta$ -neighbourhood of $F$. 
\end{proposition}
\textit{Proof: }
If $F$ can be covered by $N_{\delta}(F)$ balls of radius $\delta<1$, then $F_{\delta}$ can be covered by the concentric balls of radius $2 \delta .$ Hence,
$$
\mathcal{L}^{n}\left(F_{\delta}\right) \leq N_{\delta}(F) c_{n}(2 \delta)^{n},
$$
where $c_{n}$ is the volume of the unit ball in $\mathbb{R}^{n} .$ Taking logarithms,
$$
\frac{\log \mathcal{L}^{n}\left(F_{\delta}\right)}{-\log \delta} \leq \frac{\log 2^{n} c_{n}+n \log \delta+\log N_{\delta}(F)}{-\log \delta}
$$
so
$$
\varliminf_{\delta \rightarrow 0} \frac{\log \mathcal{L}^{n}\left(F_{\delta}\right)}{-\log \delta} \leq -n+\underline{\operatorname{dim}}_{\mathrm{B}} F
$$
with a similar inequality for the upper limits. On the other hand, if there are $N_{\delta}(F)$ disjoint balls of radius $\delta$ with centres in $F$, then by adding their volumes,
$$
N_{\delta}(F) c_{n} \delta^{n} \leq \mathcal{L}^{n}\left(F_{\delta}\right)
$$
Taking logarithms and letting $\delta \rightarrow 0$ gives the opposite inequality to the third inequality, using Equivalent definition $(\mathrm{v})$.


\textbf{\\Note: }In the context of Proposition above, box dimension is sometimes referred to as Minkowski dimension or Minkowski–Bouligand dimension.


\newpage
\subsection{Properties and Problems of Box-Counting Dimension}



\begin{proposition}\label{prop2.5} $ $
    \begin{enumerate}[a.]
        \item If $F \subset \mathbb{R}^{n}$ and $f: F \rightarrow \mathbb{R}^{m}$ is a Lipschitz transformation, that is,
        $$
        |f(x)-f(y)| \leq c|x-y| \quad(x, y \in F),
        $$
        then $\underline{\operatorname{dim}_{\mathrm{B}}} f(F) \leq \underline{\operatorname{dim}_{\mathrm{B}}} F$ and $\overline{\operatorname{dim}}_{\mathrm{B}} f(F) \leq \overline{\operatorname{dim}}_{\mathrm{B}} F$.
        
        \item If $F \subset \mathbb{R}^{n}$ and $f: F \rightarrow \mathbb{R}^{m}$ is a bi-Lipschitz transformation, that is,
        $$
        c_{1}|x-y| \leq|f(x)-f(y)| \leq c|x-y| \quad(x, y \in F),
        $$
        where $0<c_{1} \leq c<\infty$, then $\underline{\operatorname{dim}_{B}} f(F)=\underline{\operatorname{dim}}_{B} F$ and
        $\overline{\operatorname{dim}}_{\mathrm{B}} f(F)=\overline{\operatorname{dim}}_{\mathrm{B}} F .$
        
    \end{enumerate}
\end{proposition}


\newpage
\section{May 28-31}
\subsection{Exercises and Solutions}

\begin{customexercise}{1.18}
    Let $A_{1}, A_{2}, \ldots$ be a decreasing sequence of Borel subsets of $\mathbb{R}^{n}$ and let $A=\bigcap_{k=1}^{\infty} A_{k} .$ If $\mu$ is a measure on $\mathbb{R}^{n}$ with $\mu\left(A_{1}\right)<\infty$, show using (1.6) that $\mu\left(A_{k}\right) \rightarrow \mu(A)$ as $k \rightarrow \infty$.
\end{customexercise}

\begin{customsol}{1.18}
    Consider that $\{A_1\setminus A_k\}$ is an increasing sequence as $\{A_k\}$ decreasing. Then:
    $$ 
    \begin{aligned}
        &\mu\left(\bigcup_{k=1}^{\infty}\left(A_{1}\setminus A_{k}\right)\right) = \mu\left(A_{1} \setminus \bigcap_{k=1}^{\infty} A_{i}\right) = \mu(A_1) - \mu (A) \\
        =&\lim _{k \rightarrow \infty} \mu\left(A_{1} \setminus A_{k}\right) = \lim_{k\rightarrow\infty} (\mu(A_1)-\mu (A_k)) = \mu(A_1) - \lim_{k\rightarrow\infty} \mu(A_k)
    \end{aligned}$$
    As $\mu(A_1)<\infty$, $\displaystyle \lim_{k\rightarrow \infty} \mu(A_k) = \mu(\bigcap_{k=1}^{\infty} A_{i})$
\end{customsol}

\textbf{Conlusion}: For a \textbf{\textit{decreasing sequence}} $A_k$ of Borel subsets of $\mathbb{R}^n$, 
$$\displaystyle \lim_{k\rightarrow \infty} \mu(A_k) = \mu(\bigcap_{k=1}^{\infty} A_{i})$$


\begin{customexercise}{1.23}
    Let $D$ be a Borel subset of $\mathbb{R}^{n}$ and let $\mu$ be a measure 
    on $D$ with $\mu(D)<\infty$. Let $f_{k}: D \rightarrow \mathbb{R}$ be a 
    sequence of functions such that $f_{k}(x) \rightarrow f(x)$ for all $x$ 
    in $D$. Prove \underline{\textbf{Egoroff's theorem}}: that given $\varepsilon>0$ there exists 
    a Borel subset $A$ of $D$ with $\mu(D \backslash A)<\varepsilon$ such that 
    $f_{k}(x)$ converges to $f(x)$ uniformly for $x$ in $A$.
\end{customexercise}

\begin{customsol}{1.23} 
    Assume that for $k, n \in \mathbb{Z}^+$, $A_{k, n} = \{x\in D: |f_l(x) - f(x)| < 1/n, \forall l\geq k\}$(so we consider $\delta = 1/n$ here), 
    then we have $\displaystyle \bigcup_{k=1}^\infty A_{k, n} = D$ and 
    $A_{1, n}\subset A_{2, n}\subset A_{3, n}\subset\dots$. Next, by Property of measure \ref{propmeasure}:
    $$\displaystyle \mu(D) = \mu(\bigcup_{k=1}^\infty A_{k,n}) = \lim_{k\rightarrow\infty}\mu(A_{k,n})<\infty$$
    Hence, 
    $$\displaystyle \lim_{k\rightarrow\infty}\mu(D\setminus A_{k,n}) = \mu(D) - \lim_{k\rightarrow\infty} \mu(A_{k, n}) = 0$$
    Then, $\exists k^\prime \in \mathbb{N}$  s.t. whenever 
    $k\geq k^\prime$, $\mu(D\setminus A_{k,n}) < \displaystyle \frac{\epsilon}{2^n}$. 
    Next, we can construct $\displaystyle A = \bigcap_{n = 1}^\infty A_{k',n}$, which is a Borel subset of $D$ and satisfies:
    $$\mu(D\setminus A) = \mu(D\setminus \bigcap_{n=1}^\infty A_{k', n}) = \mu(\bigcup_{n=1}^\infty D\setminus A_{k',n}) = \sum_{n=1}^\infty \mu(D\setminus A_{k',n}) < \sum_{n=1}^\infty \frac{\epsilon}{2^n} = \epsilon$$
    As $\displaystyle \sum_{n=1}^{\infty} \frac{1}{2^n} = 1$, and $A$ exists for the question. 
    Finally, $\forall\delta > 0, n > 1/\delta, \forall x\in A$ where $x\in A_{k', n}$ as well, such that whenever $k\in\mathbb{N}, k>k'$, $|f_k(x) - f(x)| < 1/n < \delta\Rightarrow$ $f_k(x)$ converges to $f(x)$ uniformly for $x$ in $A$.
    
    \end{customsol}

\textbf{Note: }$\displaystyle \sum_{n=1}^{\infty} \frac{1}{2^n} = 1$ is always used to construct $\epsilon$ in analysis proofs.
 
\begin{customexercise}{1.24}
    Prove that if $\mu$ is a measure on $D$ and $f: D \rightarrow \mathbb{R}$ satisfies $f(x) \geq 0$ for all $x$ in $D$ and $\int_{D} f \mathrm{~d} \mu=0$ then $f(x)=0$ for $\mu$ -almost all $x$.
\end{customexercise}

\begin{customsol}{1.24}
Suppose $f(x)\geq\epsilon>0$ on a set $E_\epsilon\subset D$ given $\epsilon>0$, we have for $x\in D\setminus E_\epsilon, f(x)=0$ and then:
$$\begin{aligned}
    0 &= \int_D f(x) d\mu\\ &= \int_{E_\epsilon} f(x)d\mu + \int_{D\setminus E_\epsilon} f(x)d\mu\\
    &\text{As }\epsilon \chi_{E_\epsilon}(x) \text{ is a simple function and by the integral of more general functions}\\
    &\geq \int \epsilon \chi_{E_\epsilon}(x) d\mu + 0 \\ &= \epsilon\mu(E_\epsilon) + 0
\end{aligned}$$
As $\epsilon\mu(E_\mu)\leq 0$ while $\epsilon>0$, we have $\mu(E_\epsilon) = 0 \Rightarrow \mu\left(\bigcup_{\epsilon\in\mathbb{R}^+} E_\epsilon\right) = \mu\left(\{x: f(x)>0\}\right) = 0\Rightarrow f(x) = 0$ for $\mu$-a.e.. 
\end{customsol}

\textbf{Note: } 
\begin{enumerate}
    \item $f(x) = 0$ for $\mu-$a.e $\Leftrightarrow$ The set of points where $f(x)!=0$($f(x)>0$ in this case) has measure zero.
    \item $\displaystyle\int_E 1 d\mu = \mu(E)$
\end{enumerate}
\begin{definition}[Simple Function]
    Simple functions are sums of linear combination of characteristic functions, e.g. $\displaystyle f(x) = \sum a_i \chi_{A_i}(x)$
\end{definition}

\newpage
\section{May 27}
\subsection{Measeres and Mass Distributions(1.3)}

\begin{definition}[Measure]\label{measure}
    We call $\mu$ a measure on $\mathbb{R}^{n}$ if $\mu$ assigns a non-negative number, possibly $\infty$, to each subset of $\mathbb{R}^{n}$ such that\\
$
\begin{aligned}
\text { (a) }& \mu(\emptyset)=0 \\
\text { (b) }& \mu(A) \leq \mu(B) \quad \text { if } A \subset B \\
\text { (c) }&\text{if } A_{1}, A_{2}, \ldots \text{ is a countable (or finite) sequence of sets, then }\\
& \mu\left(\bigcup_{i=1}^{\infty} A_{i}\right) \leq \sum_{i=1}^{\infty} \mu\left(A_{i}\right)\\
& \text{ with equality in above, that is }\\
& \mu\left(\bigcup_{i=1}^{\infty} A_{i}\right)=\sum_{i=1}^{\infty} \mu\left(A_{i}\right), \\
&\text{ if the }A_i \text{ are disjoint Borel sets.}
\end{aligned}
$
\end{definition}

Condition (a) says that the \textit{\textbf{empty set has
zero measure}}, condition (b) says \textit{\textbf{'the larger the set, the larger the measure'}} and condition (c) says that if a set is a union of a countable number of pieces (which may overlap), then the sum of the measure of the pieces is at least equal to the measure of the whole. If \textit{\textbf{a set is decomposed into a countable number of disjoint Borel sets}}, then the \textit{\textbf{total measure of the pieces equals the measure of the whole}}.

\begin{property}[Measure]\label{propmeasure} $ $
\begin{enumerate}
    \item if $B\subset A$ $A$ and $B$ are Borel sets with $\mu(B)$ finite, 
    $$\mu(A\setminus B) = \mu(A) - \mu(B)$$ 
    as $A = B\cup(A\setminus B)$ and using Definition \ref{measure} (c).
    \item if $A_{1} \subset A_{2} \subset \cdots$ is an increasing sequence of Borel sets, then
    $$
    \mu\left(\bigcup_{i=1}^{\infty} A_{i}\right)=\lim _{i \rightarrow \infty} \mu\left(A_{i}\right) .
    $$
    as $\displaystyle\bigcup_{i=1}^{\infty} A_{i}=A_{1} \cup\left(A_{2} \backslash A_{1}\right) \cup\left(A_{3} \backslash A_{2}\right) \cup \dots$,
    $$\begin{aligned} \mu\left(\bigcup_{i=1}^{\infty} A_{i}\right) &=\mu\left(A_{1}\right)+\sum_{i=1}^{\infty}\left(\mu\left(A_{i+1}\right)-\mu\left(A_{i}\right)\right) \\ &=\mu\left(A_{1}\right)+\lim _{k \rightarrow \infty} \sum_{i=1}^{k}\left(\mu\left(A_{i+1}\right)-\mu\left(A_{i}\right)\right) \\ &=\lim _{k \rightarrow \infty} \mu\left(A_{k}\right) . \end{aligned}$$
    \item A simple extension of above is that if, for $\delta>0, A_{\delta}$ are Borel sets that are increasing as $\delta$ decreases, that is, $A_{\delta^{\prime}} \subset A_{\delta}$ for $0<\delta<\delta^{\prime}$, then
    $$
    \mu\left(\bigcup_{\delta>0} A_{\delta}\right)=\lim _{\delta \rightarrow 0} \mu\left(A_{\delta}\right) \text { . }
    $$
\end{enumerate}
\end{property}

\begin{definition}[Support of $\mu$] $ $\\
    spt $\mu$, is the smallest closed set X such that $\mu(\mathbb{R}^n\setminus X) = 0$. 
\end{definition}

By above, $x$ is in the support if and only if $\forall r > 0, \mu(B(x, r)) > 0$. We say that \textbf{\textit{$\mu$ is a measure on a set $A$ if $A$ contains the support of $\mu$}}.

\begin{definition}[Mass Distributions]
    A measure on a bounded subset of $\mathbb{R}^{n}$ for which $0<\mu\left(\mathbb{R}^{n}\right)<\infty$ will be called a mass distribution, and we think of $\mu(A)$ as the mass of the set $A$.
\end{definition}

\begin{definition}[Lebsgue Measure on $\mathbb{R}$] 
    $$\mathcal{L}^{1}(A)=\inf \left\{\sum_{i=1}^{\infty}\left(b_{i}-a_{i}\right): A \subset \bigcup_{i=1}^{\infty}\left[a_{i}, b_{i}\right]\right\}$$
\end{definition}


\newpage
\section{May 26}
\subsection{Exercises and Solutions}
\begin{customexercise}{1.12}
    Let $f, g:[0,1] \rightarrow \mathbb{R}$ be Lipschitz functions. 
    Show that the functions defined on $[0,1]$ by $f(x)+g(x)$ and 
    $f(x) g(x)$ are also Lipschitz.
\end{customexercise}

\textbf{Solution}:  
\begin{enumerate}[(i)]
    \item As $f, g$ are Lipschitz function, we have $|f(x)-f(y)| \leq c_1|x-y|$ and $|g(x)-g(y)| \leq c_2|x-y|$  where $\forall x, y \in [0, 1]$ and $c_1, c_2 \geq 0$.
    Then, $|(f(x)+g(x)) - (f(y) + g(y))| = |f(x) - f(y) +g(x) - g(y)| \leq  |f(x) - f(y)| + |g(x) - g(y)|\leq (c_1 + c_2 )\cdot |x - y|, \forall x, y\in[0, 1] $. Since $(c_1+c_2)\geq 0$, the condition is satisfied and therefore
    the functions defined on $[0,1]$ by $f(x)+g(x)$ is Lipschitz. 
    \item Consider that $|f(x) -f(0)| \leq c_1 |x| \leq c_1, x\in[0, 1]$, so we have non-negative $c_3 = |f(0)| + c_1 \geq |f(x)|$. Similarly, 
    we have non-negative $c_4 \geq |g(x)|$
    
    $|f|, |g| < 1$, $\forall x, y\in[0, 1]$\\
    $\begin{aligned}
        &|f(x)g(x) - f(y)g(y)| \\
        =& |f(x)g(x) -f(x)g(y) + f(x)g(y)- f(y)g(y)| \\
        =& |f(x)(g(x) - g(y)) + g(y)(f(x) - f(y))| \\ 
        \leq& |g(y)||(f(x) - f(y))| + |f(x)||(g(x) - g(y))| \\
        \leq& c_1 c_4 ||x-y| + c_2 c_3|x-y| \\
        \leq& (c_1c_4 +c_2c_3) |x-y|
    \end{aligned}$\\
    Since $c_1c_4 +c_2c_3 \geq 0$, the condition is satisfied and therefore
    the functions defined on $[0,1]$ by $f(x)g(x)$ is Lipschitz.
\end{enumerate}


\begin{customexercise}{1.13}
    Let $f: \mathbb{R} \rightarrow \mathbb{R}$ be differentiable 
    with $\left|f^{\prime}(x)\right| \leq c$ for all $x$. 
    Show, using the mean value theorem, that $f$ is a Lipschitz function.
\end{customexercise}

\textbf{Solution}: 
$\forall x, y \in \mathbf{R} , x \neq y$, by mean-value theorem, $\exists w \in(x, y)$ such that 
$$
\begin{aligned}
    &\frac{f(y)-f(x)}{y-x}=f^{\prime}(w) \\
    \Rightarrow  &\left|\frac{f(y)-f(x)}{y-x}\right|=\left|f^{\prime}(w)\right| \leq c \\
    \Rightarrow &|f(x)-f(y)| \leq c|x-y| \quad (x, y \in \mathbb{R})
\end{aligned}
$$

Therefore, $f$ is a Lipschitz function.


\begin{customexercise}{1.14}
    Show that every Lipschitz function 
    $f: \mathbb{R} \rightarrow \mathbb{R}$ is continuous.
\end{customexercise}

\textbf{Solution: } See proof wrote for Theorem \ref{LipschitzContinuous}.


\begin{customexercise}{1.15}
    Let $f: \mathbb{R} \rightarrow \mathbb{R}$ be given by $f(x)=x^{2}+x .$ 
    Find (i) $f^{-1}(2)$, (ii) $f^{-1}(-2)$ and (iii) $f^{-1}([2,6])$.
\end{customexercise}

\textbf{Solution}: As $f(x) = x^2 + x$, $\displaystyle x = -\frac{1}{2} \pm \frac{\sqrt{1 + 4y}}{2}$
\begin{enumerate}[(i)]
    \item $f^{-1}(2) = \{-2, 1\}$
    \item $f^{-1}(-2) = \emptyset$
    \item As $\displaystyle x = -\frac{1}{2} + \frac{\sqrt{1 + 4y}}{2}$ is increasing and $\displaystyle x = -\frac{1}{2} - \frac{\sqrt{1 + 4y}}{2}$ is decreasing while y increasing, 
     $f^{-1}([2,6]) = [-3, -2]\cup [1, 2]$
\end{enumerate}


\begin{customexercise}{1.16}
    Show that $f(x)=x^{2}$ is Lipschitz on $[0,2]$, bi-Lipschitz on 
    $[1,2]$ and not Lipschitz on $\mathbb{R}$.
\end{customexercise}

\textbf{Solution}: 
\begin{enumerate}[(i)]
    \item As $\forall x, y\in [0,2]$, $|x+y|\leq 4$, we have $|f(x)-f(y)|=\left|x^{2}-y^{2}\right|=|x+y||x-y| \leq 4|x-y|$.
    Thus, $f$ is Lipschitz on $[0,2]$.
    \item Apparently, $2|x-y|\leq|f(x) - f(y)|\leq 4|x-y|$ by above. As $f([1, 2]) = [1, 4]$, $\forall x, y \in [1, 4]$, $\displaystyle \frac{1}{\sqrt{x}+\sqrt{y}} \leq \frac{1}{2}$, we have: \\
    \(
    \begin{aligned}
        &\displaystyle \left|f^{-1}(x)-f^{-1}(y)\right|=|\sqrt{x}-\sqrt{y}|=\left|\frac{x-y}{\sqrt{x}+\sqrt{y}}\right| \leq \frac{1}{2}|x-y|\\
        \Rightarrow & \text{ so } f^{-1} \text{ is Lipschitz on } [1,4]. \\
        \Rightarrow & f \text{ is bi-Lipschitz on }[1,2].
    \end{aligned}
    \)
    \item Let $x = ky, k\in\mathbb{R}\setminus\{0\}$, then $\displaystyle \frac{|f(x) - f(y)|}{|x - y|} = \frac{|k^2y^2 - y^2|}{|ky - y|} = \left|\frac{k^2 - 1}{k-1} \right | |y|$, which is unbounded on $\mathbb{R}$. Therefore, the Lipschitz constant does not exist and $f$ is not Lipschitz on $\mathbb{R}$

\end{enumerate}


\newpage
\section{May 25}
\subsection{Basic Set Theory(1.1)}

Review and summary of some definitions and theorems:

\begin{definition}[Countable]
    An infinite set $A$ is countable if its elements 
    can be listed in the form $x_1, x_2, ...$ with every 
    element of appearing at a specific place in the list; 
    otherwise, the set is uncountable
\end{definition}

\begin{definition}[Open]
    $A \subset \mathbb{R}^n$ is open if, $\forall x\in A$, $\exists B(x, r)\in A$ where $r>0$.
\end{definition}

\begin{definition}[Closed]
    $A \subset \mathbb{R}^n$ is closed if, whenever $\{x_k\}\in A$, $x_k\rightarrow x \in \mathbb{R}^n$, then $x\in A$.
\end{definition}

\begin{definition}[Closure]\label{closure}
    $\bar{A}$ is the intersection of all the closed sets containing a set A. 
\end{definition}

\begin{definition}[Interior]\label{interior}
    $int(A)$ is the union of all open sets contained in $A$.
\end{definition}

Definition \ref{closure} and \ref{interior} shows that The \textit{closure} 
of $A$ is thought of as the \textbf{smallest closed set} containing $A$, 
and the \textit{interior} as the \textbf{largest open set} contained in $A$.

\begin{definition}[Boundary]
    $\partial A = \bar{A}\setminus int(A)$
\end{definition}

\begin{theorem}
    $x\in\partial A \Leftrightarrow \forall r > 0, B(x, r) \cap A \neq \emptyset, B(x, r) \cap A^C \neq \emptyset$ 
\end{theorem}

\begin{definition}[Dense]
    Set $B$ is a dense in $A$ if $A\subset \bar{B}$, that is, if there are points of $B$ arbitrarily close to each point of $A$.
\end{definition}

\begin{definition}[Compact]
    $A$ is compact if any collection of open sets that covers $A$ has a finite subcollection which also covers $A$.
\end{definition}

\begin{theorem}
    A compact subset of $\mathbb{R}^n$ is both closed and bounded.
\end{theorem}

\begin{theorem}
    The intersection of any collection of compact sets is compact.
\end{theorem}

\begin{definition}[Connected]
    $A\subset \mathbb{R}^n$ is connected if there not exists open sets $U$ and $V$ s.t. $A\in U\cap V$ with disjoint and nonempty $A\cap U$ and $A\cap V$.
\end{definition}

\begin{definition}[Connected Component]
Connected component of $x$ is the largest connected subset of $A$ containing a point $x$.
\end{definition}

\begin{definition}[Disconnect]
    The set $A$ is totally disconnected if the connected component of each point consists of just that point.
\end{definition}

The definition of \textit{disconnect} also can be as: $\exists$ open sets $U$ and $V$ s.t. $x\in U, y\in V$ and $A\subset U\cap V$.

\begin{definition}[Borel Set]
    Borel Sets is the smallest collection fo subsets of $\mathbb{R}^n$ with the following properties:
    \begin{enumerate}
        \item Every open set and every closed set is a Borel set.
        \item The union of every finite or countable collection of Borel sets is a Borel set, and the intersection of every finite or countable collection of Borel sets is a Borel set.
    \end{enumerate}
\end{definition}

In short, Any set that can be constructed using a sequence of countable unions or intersections starting with the open sets or closed sets will certainly be Borel.

\subsection{Functions and Limits(1.2)}

\begin{definition}[Congruence]
    The transformation $S : \mathbb{R}^n\rightarrow \mathbb{N}^n$ is congruence or isometry if it preserves distances i.e. if $|S(x)-S(y)| = |x-y|$ for $x, y\in \mathbb{R}^n$
    
\end{definition}

Special cases include \textit{translations}, which are of the form 
$S(x)=x+a$ and have the effect of shifting points parallel 
to the vector $a$, \textit{rotations} which have a centre $a$ such 
that $|S(x)-a|=|x-a|$ for all $x$ (for convenience, we also 
regard the identity transformation given by $I(x)=x$ as a 
rotation) and \textit{reflections}, which maps points to their mirror 
images in some $(n-1)$-dimensional plane. A congruence that 
may be achieved by a combination of a rotation and a translation, 
that is, does not involve reflection, is called a \textit{rigid motion} or 
\textit{direct congruence}. A transformation 
$S: \mathbb{R}^{n} \rightarrow \mathbb{R}^{n}$ is a \textit{similarity} of 
\textit{ratio} or \textit{scale} $c>0$ if $|S(x)-S(y)|=c|x-y|$ for all $x, y$ 
in $\mathbb{R}^{n} .$ A similarity transforms sets into geometrically 
similar ones with all lengths multiplied by the factor $c$.

\begin{definition}[Linear Transformation]
    A transformation $T : \mathbb{R}^n \rightarrow \mathbb{R}^n$ is linear 
    if $\forall x, y \in \mathbb{R}^n, T(x+y) = T(x) + T(y)$ and 
    $T(\lambda x) = \lambda T(x), \lambda \in\mathbb{R}$ 
\end{definition}

Such a linear transformation is \textit{non-singular} if $T(x)=0$ if and only if $x=0 .$ If 
$S: \mathbb{R}^{n} \rightarrow \mathbb{R}^{n}$ is of the form $S(x)=T(x)+a$, 
where $T$ is a non-singular linear transformation and $a$ is a vector in 
$\mathbb{R}^{n}$, then $S$ is called an \textit{affine transformation}\textit{ or an 
affinity}. An affinity may be thought of as a shearing transformation; 
its contracting or expanding effect need not be the same in every direction. 
However, if $T$ is orthonormal, then $s$ is a congruence, and if $T$ is a 
scalar multiple of an orthonormal transformation, then $T$ is a similarity.

\begin{definition}[Hölder Function]
    A function $f: X \rightarrow Y$ is called a Hölder function of exponent $\alpha$ if
$$
|f(x)-f(y)| \leq c|x-y|^{\alpha} \quad(x, y \in X)
$$
    for some constant $c\geq 0$.
\end{definition}

\begin{definition}[Libschitz Function]
    The function $f$ is called Lipschitz if
    $$
    |f(x)-f(y)| \leq c|x-y| \quad(x, y \in X)
    $$
    and bi-Lipschitz if 
    $$
    c_{1}|x-y| \leq |f(x)-f(y)| \leq c_{2}|x-y| \quad(x, y \in X)
    $$
    for $0<c_{1} \leq c_{2}<\infty$, in which case both $f$ and 
    $f^{-1}: f(X) \rightarrow X$ are Lipschitz functions.     
\end{definition}

\begin{definition}[Lower Limit]
    $$\displaystyle \lowlim_{x \to 0} f(x) \equiv \lim_{r\to 0}(inf\{f(x):0<x<r\})$$
\end{definition}

\textbf{Note:} $inf\{f(x):0<x<r\}$ is either $-\infty$ for all positive $r$ or else increases as $r$ decreases, 
$\lowlim_{x \to 0} f(x)$ always exists.

\begin{definition}[Upper Limit]
    $$\displaystyle \uplim_{x \to 0} f(x) \equiv  \lim_{r\to 0}(sup\{f(x):0<x<r\})$$
\end{definition}

\textbf{Note:} The lower and upper limits exist as real numbers or $-\infty$ or $\infty$ for 
every function $f$ and are indicative of the variation of $f$ for x close to 0, shown in Figure \ref{fig:lowerupperlimit}.

\begin{figure}[H]
    \centering
    \includegraphics[width=.66\textwidth]{images/limit1.png}
    \caption{The upper and lower limits of a function.}
    \label{fig:lowerupperlimit}
\end{figure}

We write $f(x) \sim g(x)$ to mean that $f(x) / g(x) \rightarrow 1$ as $x \rightarrow 0$.

\begin{theorem}[Lipschitz functions are continuous]\label{LipschitzContinuous}
\end{theorem}

\textit{Proof}: Assum that the function $f: X \rightarrow Y$ is a Lipschitz function s.t. 
$|f(x)-f(y)| \leq c|x-y| (x, y \in X)$ for some constant $c\geq 0$.
Then, $\forall \epsilon > 0$, let $\displaystyle \delta = \frac{\epsilon}{c}$, and we have
$\forall x, y\in X, |x-y| < \delta \Rightarrow \displaystyle |x-y| < \frac{\epsilon}{c} \Rightarrow |f(x) - f(y)| \leq c |x-y| \leq c\cdot \frac{\epsilon}{c} = \epsilon \Rightarrow$ Lipschitz functions are continuous.

\begin{definition}[Homeomorphism]
    If $f: X \rightarrow Y$ is a continuous bijection with continuous 
    inverse $f^{-1}: Y \rightarrow X$, then $f$ is called a homeomorphism, 
    and $X$ and $Y$ are termed homeomorphic sets. 
\end{definition}
\begin{corollary}
    Congruences, similarities 
    and affine transformations on $\mathbb{R}^{n}$ are examples of homeomorphisms.
\end{corollary}

\begin{definition}[Differentiable]
    If $f: \mathbb{R}^{n} \rightarrow \mathbb{R}^{n}$, we say that $f$ is 
    differentiable at $x$ and has derivative given by the linear mapping 
    $f^{\prime}(x): \mathbb{R}^{n} \rightarrow \mathbb{R}^{n}$ if
$$
\lim _{h \rightarrow 0} \frac{\left|f(x+h)-f(x)-f^{\prime}(x) h\right|}{|h|}=0 .
$$
\end{definition}

\begin{definition}[Pointwise Convergence]
    For a sequence of functions: $f_k : X\rightarrow Y$ where $X$ and $Y$ are subsets 
    of Euclidean spaces. $f_k$ converge pointwise to a function $f:X\rightarrow Y$ if 
    $f_k(x)\rightarrow f(x)$ as $k\rightarrow \infty$.
\end{definition}

\begin{definition}[Unifrom Convergence]
    For a sequence of functions: $f_k : X\rightarrow Y$ where $X$ and $Y$ are subsets 
    of Euclidean spaces. $f_k$ converge uniformly to a function $f:X\rightarrow Y$ if 
    $sup_{x\in X} |f_k(x)- f(x)| \rightarrow 0$ as $k\rightarrow \infty$.
\end{definition}

\textbf{Note}: Uniform convergence is a stronger property than pointwise convergence i.e. Uniform convergence implies pointwise convergence, but not the other way around

\begin{definition}[Another Definition of Pointwise Convergence]
For each $x\in D$, $\forall\delta > 0$, $\exists k_{x, \delta}>0$, s.t. whenever $k > k_{x, \delta}$, $|f_k(x)-f(x)| < \delta$.
\end{definition}

\begin{definition}[Another Definition of Uniform Convergence]
$\forall\delta >0$, $\exists k_\delta>0$ s.t. whenever $k>k_\delta$, $|f_k(x) - f(x)|<\delta$.
\end{definition}

\textbf{Note}: the main difference between pointwise and uniform convergence is that pointwise convergence is for each $x$ in the domain, whereas uniform convergence is for all $x$ in domain. And this is also the reason why $\sup$ shown in the definition in the textbook.



\begin{theorem}
    If the functions $f_k$ are continuous and converge uniformly to $f$, then $f$ is continuous. 
\end{theorem}


\begin{theorem}[Logarithms]
    Apparently, $a^c = b^{c\log a / \log b}$
\end{theorem}
\end{document}