\documentclass[12pt, a4paper]{article}


\usepackage{hyperref}
\hypersetup{
    colorlinks,
    citecolor=blue,
    filecolor=blue,
    linkcolor=blue,
    urlcolor=blue
}

\usepackage{amsmath}
% \usepackage{amsthm}
% \theoremstyle{definition}
\newtheorem{definition}{Definition}[section]
\newtheorem{theorem}{Theorem}[section]
\usepackage{graphicx}
\usepackage{amsfonts}




\title{\textbf{Graph Rectifiability Summer Research with Lisa Naples} \\ [0.5cm] \sffamily{Daily Report}}

\author{\sffamily{By} \\[.2cm] \textbf{\textit{Charles Zhang}}  \\ [.5cm]}

\date{Summer 2021}

\begin{document}

\maketitle

\newpage
{
    \hypersetup{linkcolor=black}
    \tableofcontents
}

\newpage

\section{May 25}
\subsection{Basic Set Theory}

\begin{definition}[Countable]
    An infinite set $A$ is countable if its elements 
    can be listed in the form $x_1, x_2, ...$ with every 
    element of appearing at a specific place in the list; 
    otherwise, the set is uncountable
\end{definition}

\begin{definition}[Open]
    $A \subset \mathbb{R}^n$ is open if, $\forall x\in A$, $\exists B(x, r)\in A$ where $r>0$.
\end{definition}

\begin{definition}[Closed]
    $A \subset \mathbb{R}^n$ is closed if, whenever $\{x_k\}\in A$, $x_k\rightarrow x \in \mathbb{R}^n$, then $x\in A$.
\end{definition}

\begin{definition}[Closure]\label{closure}
    $\bar{A}$ is the intersection of all the closed sets containing a set A. 
\end{definition}

\begin{definition}[Interior]\label{interior}
    $int(A)$ is the union of all open sets contained in $A$.
\end{definition}

Definition \ref{closure} and \ref{interior} shows that The \textit{closure} 
of $A$ is thought of as the \textbf{smallest closed set} containing $A$, 
and the \textit{interior} as the \textbf{largest open set} contained in $A$.

\begin{definition}[Boundary]
    $\partial A = \bar{A}\setminus int(A)$
\end{definition}

\begin{theorem}
    $x\in\partial A \Leftrightarrow \forall r > 0, B(x, r) \cap A \neq \emptyset, B(x, r) \cap A^C \neq \emptyset$ 
\end{theorem}

\begin{definition}[Dense]
    Set $B$ is a dense in $A$ if $A\subset \bar{B}$, that is, if there are points of $B$ arbitrarily close to each point of $A$.
\end{definition}

\begin{definition}[Compact]
    $A$ is compact if any collection of open sets that covers $A$ has a finite subcollection which also covers $A$.
\end{definition}

\begin{theorem}
    A compact subset of $\mathbb{R}^n$ is both closed and bounded.
\end{theorem}

\begin{theorem}
    The intersection of any collection of compact sets is compact.
\end{theorem}

\begin{definition}[Connected]
    $A\subset \mathbb{R}^n$ is connected if there not exists open sets $U$ and $V$ s.t. $A\in U\cap V$ with disjoint and nonempty $A\cap U$ and $A\cap V$.
\end{definition}

\begin{definition}[Connected Component]
    The connected component of $x$ is the largest connected subset of $A$ containing a point $x$.
\end{definition}

\begin{definition}[Disconnect]
    The set $A$ is totally disconnected if the connected component of each point consists of just that point.
\end{definition}

The definition of \textit{disconnect} also can be as: $\exists$ open sets $U$ and $V$ s.t. $x\in U, y\in V$ and $A\subset U\cap V$.

\begin{definition}[Borel Set]
    Borel Sets is the smallest collection fo subsets of $\mathbb{R}^n$ with the following properties:
    \begin{enumerate}
        \item Every open set and every closed set is a Borel set.
        \item The union of every finite or countable collection of Borel sets is a Borel set, and the intersection of every finite or countable collection of Borel sets is a Borel set.
    \end{enumerate}
\end{definition}

In short, Any set that can be constructed using a sequence of countable unions or intersections starting with the open sets or closed sets will certainly be Borel.

\subsection{Functions and Limits}

\begin{definition}[Congruence]
    The transformation $S : \mathbb{R}^n\rightarrow \mathbb{N}^n$ is congruence or isometry if it preserves distances i.e. if $|S(x)-S(y)| = |x-y|$ for $x, y\in \mathbb{R}^n$
    
\end{definition}

Special cases include \textit{translations}, which are of the form 
$S(x)=x+a$ and have the effect of shifting points parallel 
to the vector $a$, \textit{rotations} which have a centre $a$ such 
that $|S(x)-a|=|x-a|$ for all $x$ (for convenience, we also 
regard the identity transformation given by $I(x)=x$ as a 
rotation) and \textit{reflections}, which maps points to their mirror 
images in some $(n-1)$ -dimensional plane. A congruence that 
may be achieved by a combination of a rotation and a translation, 
that is, does not involve reflection, is called a \textit{rigid motion} or 
\textit{direct congruence}. A transformation 
$S: \mathbb{R}^{n} \rightarrow \mathbb{R}^{n}$ is a \textit{similarity} of 
\textit{ratio} or \textit{scale} $c>0$ if $|S(x)-S(y)|=c|x-y|$ for all $x, y$ 
in $\mathbb{R}^{n} .$ A similarity transforms sets into geometrically 
similar ones with all lengths multiplied by the factor $c$.

\begin{definition}
    
\end{definition}
\end{document}